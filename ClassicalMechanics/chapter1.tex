\documentclass{article}
\usepackage{graphicx}
\usepackage{amsmath}


\title{Chapter 1}
\author{Bhargav Annem}
\date{May 2023}

\begin{document}
\maketitle

\section{Derivations}
Let's derive Newton's Second Law for rotational kinematics.
\begin{equation}
    \begin{aligned}
         & F = ma                              \\
         & F = F_r \hat{r} + F_\phi \hat{\phi} \\
    \end{aligned}
\end{equation}
First, the conversions for rectangular to angular coordinates are as follows:
\begin{equation}
    \begin{aligned}
         & x = r\cos(\phi)          \\
         & y = r\sin(\phi)          \\
         & \phi = \tan(\frac{y}{x}) \\
         & r = \sqrt{x^2+y^2}       \\
    \end{aligned}
\end{equation}

Also,
\begin{equation}
    \begin{aligned}
         & \Delta r=\Delta \phi \hat{\phi}                   \\
         & \Delta r = \dot{\phi} \Delta t \hat{\phi}         \\
         & \frac{\Delta r}{\Delta t} = \dot{\phi} \hat{\phi} \\
         & \frac{d\hat{r}}{dt} = \dot{\phi} \hat{\phi}       \\
    \end{aligned}
\end{equation}

To prove this last statement more rigorously, let's decompose $\vec{r}$ into Cartesian components.

\[\vec{r} = r\cos(\phi)\hat{x} + r\sin(\phi)\hat{y}\]
Then,
\[\frac{d\vec{r}}{dr} = \cos(\phi)\hat{x} + \sin(\phi)\hat{y}\]
since
\[\frac{d\vec{r}}{dr} = \cos(\phi)\hat{x} + \sin(\phi)\hat{y}\]
and
\[|\frac{d\vec{r}}{dr}| = \sqrt{ \cos(\phi)^2 + \sin(\phi)^2 } = 1 \]
\[ \hat{r} = \cos(\phi)\hat{x} + \sin(\phi)\hat{y}\]

Similarly, solving for $\frac{d\vec{r}}{d\phi}$ gives us the following:

\[\frac{d\vec{r}}{d\phi} = r(-\sin(\phi)\hat{x} + \cos(\phi)\hat{y})\]
and
\[|\frac{d\vec{r}}{d\phi}| = r\sqrt{(-\sin(\phi)^2) + \cos(\phi)^2} = 1\]
so
\[\hat{\phi} = -\sin(\phi)\hat{x} + \cos(\phi)\hat{y}\]

Initially, we established that
\[\vec{r} = \it{r}\hat{r}\]
Then,
\begin{equation}
    \begin{aligned}
        \vec{v} = \frac{\vec{dr}}{dt} & = \frac{d}{dt}(\it{r}\hat{r})                                                      \\
                                      & = r\frac{d\hat{r}}{dt} + \hat{r}\dot{r} + \it{\dot{r}}\hat{r}                      \\
                                      & = r(\frac{-d\phi}{dt}\sin(\phi) + \frac{d\phi}{dt}cos(\phi)) + \it{\dot{r}}\hat{r} \\
                                      & = r(\frac{d\phi}{dt}(-\sin(\phi) + cos(\phi))) + \it{\dot{r}}\hat{r}               \\
                                      & = r(\dot{\phi}(-\sin(\phi) + cos(\phi))) + \it{\dot{r}}\hat{r}                     \\
                                      & = r\dot{\phi}\hat{\phi} + \it{\dot{r}}\hat{r}                                      \\
    \end{aligned}
\end{equation}

Then,
\begin{equation}
    \begin{aligned}
        \frac{d^2\vec{r}}{dt^2} = r(\frac{d\dot{\phi}\hat{\phi}}{dt}) + \dot{\phi}\hat{\phi}\dot{r} + \dot{r}\frac{d\hat{r}}{dt} + \hat{r}\ddot{r}                         \\
        \frac{d^2\vec{r}}{dt^2} = r(\ddot{\phi}\hat{\phi} + \dot{\phi}\frac{d\hat{\phi}}{dt}) + \dot{\phi}\hat{\phi}\dot{r} + \dot{r}\frac{d\hat{r}}{dt} + \hat{r}\ddot{r} \\
    \end{aligned}
\end{equation}
And since,
\begin{equation}
    \begin{aligned}
        \frac{d\hat{\phi}}{dt} & = -\dot{\phi}\cos(\phi)\hat{x} - \dot{\phi}\sin(\phi)\hat{y}   \\
        \frac{d\hat{\phi}}{dt} & = -\dot{\phi}(\cos(\phi)\hat{x} + \sin(\phi)\hat{y})           \\
        \frac{d\hat{\phi}}{dt} & = -\dot{\phi}\hat{r}                                           \\
        \frac{d\hat{r}}{dt}    & = -\dot{\phi}(\sin(\phi)\hat{x} - \dot{\phi}\cos(\phi)\hat{y}) \\
        \frac{d\hat{r}}{dt}    & = \dot{\phi}\hat{\phi}                                         \\
    \end{aligned}
\end{equation}
We end with,
\begin{equation}
    \begin{aligned}
        \frac{d^2\vec{r}}{dt^2} = r(\ddot{\phi}\hat{\phi} - \dot{\phi}\dot{\phi}\hat{r}) + \dot{\phi}\hat{\phi}\dot{r} + \dot{r}\dot{\phi}\hat{\phi} + \ddot{r}\hat{r} \\
        \vec{a} = \frac{d^2\vec{r}}{dt^2} = (\ddot{r} - r\dot{\phi}^2\hat{r})\hat{r} + (2r\dot{\phi} + \ddot{\phi})\hat{\phi}                                          \\
    \end{aligned}
\end{equation}

From here, we conclude with:
\begin{equation}
    \begin{aligned}
        \boxed{\vec{F} = m\vec{a} = m(\ddot{r} - r\dot{\phi}^2\hat{r})\hat{r} + m(2r\dot{\phi} + \ddot{\phi})\hat{\phi}} \\
    \end{aligned}
\end{equation}
\end{document}