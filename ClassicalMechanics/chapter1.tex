\documentclass{article}
\usepackage{graphicx}
\usepackage{amsmath}

\newcounter{problem}
\newcounter{solution}

\newcommand\Problem{%
  \stepcounter{problem}%
  \textbf{\theproblem.}~%
  \setcounter{solution}{0}%
}

\newcommand\TheSolution{%
  \textbf{Solution:}\\%
}

\newcommand\ASolution{%
  \stepcounter{solution}%
  \textbf{Solution \thesolution:}\\%
}
\parindent 0in
\parskip 1em

\title{Chapter 1}
\author{Bhargav Annem}
\date{May 2023}

\begin{document}
\maketitle

\section{Derivations}
Let's derive Newton's Second Law for rotational kinematics.
\begin{equation}
    \begin{aligned}
         & F = ma                              \\
         & F = F_r \hat{r} + F_\phi \hat{\phi} \\
    \end{aligned}
\end{equation}
First, the conversions for rectangular to angular coordinates are as follows:
\begin{equation}
    \begin{aligned}
         & x = r\cos(\phi)          \\
         & y = r\sin(\phi)          \\
         & \phi = \tan(\frac{y}{x}) \\
         & r = \sqrt{x^2+y^2}       \\
    \end{aligned}
\end{equation}

Also,
\begin{equation}
    \begin{aligned}
         & \Delta r=\Delta \phi \hat{\phi}                   \\
         & \Delta r = \dot{\phi} \Delta t \hat{\phi}         \\
         & \frac{\Delta r}{\Delta t} = \dot{\phi} \hat{\phi} \\
         & \frac{d\hat{r}}{dt} = \dot{\phi} \hat{\phi}       \\
    \end{aligned}
\end{equation}

To prove this last statement more rigorously, let's decompose $\vec{r}$ into Cartesian components.

\[\vec{r} = r\cos(\phi)\hat{x} + r\sin(\phi)\hat{y}\]
Then,
\[\frac{d\vec{r}}{dr} = \cos(\phi)\hat{x} + \sin(\phi)\hat{y}\]
since
\[\frac{d\vec{r}}{dr} = \cos(\phi)\hat{x} + \sin(\phi)\hat{y}\]
and
\[|\frac{d\vec{r}}{dr}| = \sqrt{ \cos(\phi)^2 + \sin(\phi)^2 } = 1 \]
\[ \hat{r} = \cos(\phi)\hat{x} + \sin(\phi)\hat{y}\]

Similarly, solving for $\frac{d\vec{r}}{d\phi}$ gives us the following:

\[\frac{d\vec{r}}{d\phi} = r(-\sin(\phi)\hat{x} + \cos(\phi)\hat{y})\]
and
\[|\frac{d\vec{r}}{d\phi}| = r\sqrt{(-\sin(\phi)^2) + \cos(\phi)^2} = 1\]
so
\[\hat{\phi} = -\sin(\phi)\hat{x} + \cos(\phi)\hat{y}\]

Initially, we established that
\[\vec{r} = \it{r}\hat{r}\]
Then,
\begin{equation}
    \begin{aligned}
        \vec{v} = \frac{\vec{dr}}{dt} & = \frac{d}{dt}(\it{r}\hat{r})                                                      \\
                                      & = r\frac{d\hat{r}}{dt} + \hat{r}\dot{r} + \it{\dot{r}}\hat{r}                      \\
                                      & = r(\frac{-d\phi}{dt}\sin(\phi) + \frac{d\phi}{dt}cos(\phi)) + \it{\dot{r}}\hat{r} \\
                                      & = r(\frac{d\phi}{dt}(-\sin(\phi) + cos(\phi))) + \it{\dot{r}}\hat{r}               \\
                                      & = r(\dot{\phi}(-\sin(\phi) + cos(\phi))) + \it{\dot{r}}\hat{r}                     \\
                                      & = r\dot{\phi}\hat{\phi} + \it{\dot{r}}\hat{r}                                      \\
    \end{aligned}
\end{equation}

Then,
\begin{equation}
    \begin{aligned}
        \frac{d^2\vec{r}}{dt^2} = r(\frac{d\dot{\phi}\hat{\phi}}{dt}) + \dot{\phi}\hat{\phi}\dot{r} + \dot{r}\frac{d\hat{r}}{dt} + \hat{r}\ddot{r}                         \\
        \frac{d^2\vec{r}}{dt^2} = r(\ddot{\phi}\hat{\phi} + \dot{\phi}\frac{d\hat{\phi}}{dt}) + \dot{\phi}\hat{\phi}\dot{r} + \dot{r}\frac{d\hat{r}}{dt} + \hat{r}\ddot{r} \\
    \end{aligned}
\end{equation}
And since,
\begin{equation}
    \begin{aligned}
        \frac{d\hat{\phi}}{dt} & = -\dot{\phi}\cos(\phi)\hat{x} - \dot{\phi}\sin(\phi)\hat{y}   \\
        \frac{d\hat{\phi}}{dt} & = -\dot{\phi}(\cos(\phi)\hat{x} + \sin(\phi)\hat{y})           \\
        \frac{d\hat{\phi}}{dt} & = -\dot{\phi}\hat{r}                                           \\
        \frac{d\hat{r}}{dt}    & = -\dot{\phi}(\sin(\phi)\hat{x} - \dot{\phi}\cos(\phi)\hat{y}) \\
        \frac{d\hat{r}}{dt}    & = \dot{\phi}\hat{\phi}                                         \\
    \end{aligned}
\end{equation}
We end with,
\begin{equation}
    \begin{aligned}
        \frac{d^2\vec{r}}{dt^2} = r(\ddot{\phi}\hat{\phi} - \dot{\phi}\dot{\phi}\hat{r}) + \dot{\phi}\hat{\phi}\dot{r} + \dot{r}\dot{\phi}\hat{\phi} + \ddot{r}\hat{r} \\
        \vec{a} = \frac{d^2\vec{r}}{dt^2} = (\ddot{r} - r\dot{\phi}^2\hat{r})\hat{r} + (2r\dot{\phi} + \ddot{\phi})\hat{\phi}                                          \\
    \end{aligned}
\end{equation}

From here, we conclude with:
\begin{equation}
    \begin{aligned}
        \boxed{\vec{F} = m\vec{a} = m(\ddot{r} - r\dot{\phi}^2\hat{r})\hat{r} + m(2r\dot{\phi} + \ddot{\phi})\hat{\phi}} \\
    \end{aligned}
\end{equation}

\section{Problem Solutions}
\Problem 1.1 Given the two vectors $b = \hat{x} + \hat{y}$ and $c = \hat{x} + \hat{z}$ find $b + c$, $5b + 2c$, $b \cdot c$, and $b \times c$

\TheSolution {
    \begin{equation}
        \begin{aligned}
            b + c      & = 2\hat{x} + \hat{y} + \hat{z}   \\
            5b + 2c    & = 7\hat{x} + 5\hat{y} + 2\hat{z} \\
            b \cdot c  & = 1                              \\
            b \times c & = \begin{cases}
                               1, 1, 0 \\
                               1, 0, 1
                           \end{cases}                   \\
                       & = <1, 1, -1>                     \\
                       & = \hat{x} - \hat{y} - \hat{z}
        \end{aligned}
    \end{equation}
}

\Problem 1.3 By applying Pythagoras's theorem (the usual two-dimensional version) twice over, prove that the length $\it{r}$ of a three-dimensional vector $r = (x, y, z)$ statisfies $r^2 = x^2 + y^2 + z^2$

\TheSolution
\begin{equation}
    \begin{aligned}
        h^2 & = x^2 + y^2       \\
        r^2 & = h^2 + z^2       \\
        r^2 & = x^2 + y^2 + z^2 \\
    \end{aligned}
\end{equation}

\Problem Find the angle between a body diagonal of a cube and any of its face diagonals. [$\it{Hint:}$ Choose a cube with side 1 adn with one corner at $\it{O}$ and the opposite corner at the point (1, 1, 1). Write down teh vector that represents a body diagonal and another that represents a face diagonal, and tehn find teh angle between them as in Problem 1.4].

\TheSolution
\begin{equation}
    \begin{aligned}
        f_{body}                & = <1,1,1>                            \\
        f_{face}                & = <1,1,0>                            \\
        f_{body} \cdot f_{face} & = 2 = |f_{body}||f_{face}|\cos(\phi) \\
        \sqrt3\sqrt2\cos(\phi)  & = 2                                  \\
        \phi                    & = \arccos(\frac{2}{\sqrt3\sqrt2})
    \end{aligned}
\end{equation}
\begin{center}
    \boxed{35.26\deg}
\end{center}

\Problem Prove that the two definitions of the scalar product $r \cdot s$ as $rs\cos(\phi)$ and $\sum r_i s_i$ are equal. One way to do this is to choose your x-axis along the direction of $\boldsymbol{r}$

\TheSolution
If $r$ lays along $x$, then $\cos(\phi) = 1$ since $\phi = 0$. Since $\sum r_i s_i = \sum r_i \cdot \sum s_i = rs$, the two statements are equivalent. The longer way is to use the law of cosines to expand $\cos(\phi)$ in terms of r and s, which will give you an expression that evaluates to a summation.

\Problem
In elementary trigonometry, you probably learned the law of cosines for a triangle of sides $a, b, c$ that $c^2 = a^2 + b^2 - 2ab\cos(\phi)$ where $\phi$ is the angle between teh sides $a$ and $b$. Show that the law of cosines is an immediate consequence of the identity $(a + b)^2 = a^2 + b^2 + 2 a \cdot b$.

\TheSolution
\begin{equation}
    \begin{aligned}
        2a \cdot b & = 2ab\cos(\phi) \\
    \end{aligned}
\end{equation}
Since $\phi$ represents the angle between a and b (which is the external angle of the triangle $\pi - \phi$), then $\cos(\phi) \rightarrow -\cos(\phi)$. Let $c = a + b$, then we get $c^2 = a^2 + b^2 - 2ab\cos(\phi)$.

\Problem
The position of a moving particle is given as a function fo time $t$ to be
\[r(t) = \hat{x}b\cos(\omega t) + \hat{y}c\sin(\omega t) + \hat{z}v_o t\]
where $b, c, \text{and } \omega$ are constants. Describe the particle's orbit.

\TheSolution
The $\hat{z}v_o t$ part will cause the particle to move upwards continuously, but the two trigonometric functions of different amplitudes will create an elliptical orbit.

\Problem
Let $u$ be an arbitrary fixed unit vector and show that any vector $b$ satisfies
\[b^2 = (u \cdot b)^2 + (u \times b)^2\]

\TheSolution
\begin{equation}
    \begin{aligned}
        b^2 & = (ub\cos(\phi))^2 + (ub\sin(\phi))^2 \\
        b^2 & = (ub)^2(\cos(\phi)^2 + \sin(\phi))^2 \\
        b^2 & = u^2 b^2                             \\
    \end{aligned}
\end{equation}

Since $|u| = 1$ because it is a unit vector, we get $b^2 = b^2$.

\Problem
Show that the definition of the cross product is equivalent to the elementary definition that $r \times s$ is perpendicular to both $r$ and $s$ with magnitude $rs\sin(\phi)$ and direction given by the right-hand rule.

\TheSolution
Let $\vec{v} = (v_1, v_2, v_3)$ and $\vec{s} = (s_1, s_2, s_3)$, then:
\begin{equation}
    \begin{aligned}

    \end{aligned}
\end{equation}
\end{document}