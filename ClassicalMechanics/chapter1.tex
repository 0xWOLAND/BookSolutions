\documentclass{article}
\usepackage{graphicx}
\usepackage{amsmath}
\usepackage{bm}

\newcounter{solution}

\newcommand\Problem{%
    \subsubsection{}%
}

\newcommand\TheSolution{%
  \textbf{Solution:}\\%
}

\newcommand\ASolution{%
  \stepcounter{solution}%
  \textbf{Solution \thesolution:}\\%
}
\parindent 0in
\parskip 1em

\title{Chapter 1}
\author{Bhargav Annem}
\date{May 2023}

\begin{document}
\maketitle

\section{Derivations}
Let's derive Newton's Second Law for rotational kinematics.
\begin{equation}
    \begin{aligned}
         & F = ma                              \\
         & F = F_r \hat{r} + F_\phi \hat{\phi} \\
    \end{aligned}
\end{equation}
First, the conversions for rectangular to angular coordinates are as follows:
\begin{equation}
    \begin{aligned}
         & x = r\cos(\phi)          \\
         & y = r\sin(\phi)          \\
         & \phi = \tan(\frac{y}{x}) \\
         & r = \sqrt{x^2+y^2}       \\
    \end{aligned}
\end{equation}

Also,
\begin{equation}
    \begin{aligned}
         & \Delta r=\Delta \phi \hat{\phi}                   \\
         & \Delta r = \dot{\phi} \Delta t \hat{\phi}         \\
         & \frac{\Delta r}{\Delta t} = \dot{\phi} \hat{\phi} \\
         & \frac{d\hat{r}}{dt} = \dot{\phi} \hat{\phi}       \\
    \end{aligned}
\end{equation}

To prove this last statement more rigorously, let's decompose $\vec{r}$ into Cartesian components.

\[\vec{r} = r\cos(\phi)\hat{x} + r\sin(\phi)\hat{y}\]
Then,
\[\frac{d\vec{r}}{dr} = \cos(\phi)\hat{x} + \sin(\phi)\hat{y}\]
since
\[\frac{d\vec{r}}{dr} = \cos(\phi)\hat{x} + \sin(\phi)\hat{y}\]
and
\[|\frac{d\vec{r}}{dr}| = \sqrt{ \cos(\phi)^2 + \sin(\phi)^2 } = 1 \]
\[ \hat{r} = \cos(\phi)\hat{x} + \sin(\phi)\hat{y}\]

Similarly, solving for $\frac{d\vec{r}}{d\phi}$ gives us the following:

\[\frac{d\vec{r}}{d\phi} = r(-\sin(\phi)\hat{x} + \cos(\phi)\hat{y})\]
and
\[|\frac{d\vec{r}}{d\phi}| = r\sqrt{(-\sin(\phi)^2) + \cos(\phi)^2} = 1\]
so
\[\hat{\phi} = -\sin(\phi)\hat{x} + \cos(\phi)\hat{y}\]

Initially, we established that
\[\vec{r} = \it{r}\hat{r}\]
Then,
\begin{equation}
    \begin{aligned}
        \vec{v} = \frac{\vec{dr}}{dt} & = \frac{d}{dt}(\it{r}\hat{r})                                                      \\
                                      & = r\frac{d\hat{r}}{dt} + \hat{r}\dot{r} + \it{\dot{r}}\hat{r}                      \\
                                      & = r(\frac{-d\phi}{dt}\sin(\phi) + \frac{d\phi}{dt}cos(\phi)) + \it{\dot{r}}\hat{r} \\
                                      & = r(\frac{d\phi}{dt}(-\sin(\phi) + cos(\phi))) + \it{\dot{r}}\hat{r}               \\
                                      & = r(\dot{\phi}(-\sin(\phi) + cos(\phi))) + \it{\dot{r}}\hat{r}                     \\
                                      & = r\dot{\phi}\hat{\phi} + \it{\dot{r}}\hat{r}                                      \\
    \end{aligned}
\end{equation}

Then,
\begin{equation}
    \begin{aligned}
        \frac{d^2\vec{r}}{dt^2} = r(\frac{d\dot{\phi}\hat{\phi}}{dt}) + \dot{\phi}\hat{\phi}\dot{r} + \dot{r}\frac{d\hat{r}}{dt} + \hat{r}\ddot{r}                         \\
        \frac{d^2\vec{r}}{dt^2} = r(\ddot{\phi}\hat{\phi} + \dot{\phi}\frac{d\hat{\phi}}{dt}) + \dot{\phi}\hat{\phi}\dot{r} + \dot{r}\frac{d\hat{r}}{dt} + \hat{r}\ddot{r} \\
    \end{aligned}
\end{equation}
And since,
\begin{equation}
    \begin{aligned}
        \frac{d\hat{\phi}}{dt} & = -\dot{\phi}\cos(\phi)\hat{x} - \dot{\phi}\sin(\phi)\hat{y}   \\
        \frac{d\hat{\phi}}{dt} & = -\dot{\phi}(\cos(\phi)\hat{x} + \sin(\phi)\hat{y})           \\
        \frac{d\hat{\phi}}{dt} & = -\dot{\phi}\hat{r}                                           \\
        \frac{d\hat{r}}{dt}    & = -\dot{\phi}(\sin(\phi)\hat{x} - \dot{\phi}\cos(\phi)\hat{y}) \\
        \frac{d\hat{r}}{dt}    & = \dot{\phi}\hat{\phi}                                         \\
    \end{aligned}
\end{equation}
We end with,
\begin{equation}
    \begin{aligned}
        \frac{d^2\vec{r}}{dt^2} = r(\ddot{\phi}\hat{\phi} - \dot{\phi}\dot{\phi}\hat{r}) + \dot{\phi}\hat{\phi}\dot{r} + \dot{r}\dot{\phi}\hat{\phi} + \ddot{r}\hat{r} \\
        \vec{a} = \frac{d^2\vec{r}}{dt^2} = (\ddot{r} - r\dot{\phi}^2\hat{r})\hat{r} + (2r\dot{\phi} + \ddot{\phi})\hat{\phi}                                          \\
    \end{aligned}
\end{equation}

From here, we conclude with:
\begin{equation}
    \begin{aligned}
        \boxed{\vec{F} = m\vec{a} = m(\ddot{r} - r\dot{\phi}^2\hat{r})\hat{r} + m(2r\dot{\phi} + \ddot{\phi})\hat{\phi}} \\
    \end{aligned}
\end{equation}

\section{Problem Solutions}
\subsection{1.1-1.3 Space and Time}
\Problem 1.1 Given the two vectors $b = \hat{x} + \hat{y}$ and $c = \hat{x} + \hat{z}$ find $b + c$, $5b + 2c$, $b \cdot c$, and $b \times c$

\TheSolution {
    \begin{equation}
        \begin{aligned}
            b + c      & = 2\hat{x} + \hat{y} + \hat{z}   \\
            5b + 2c    & = 7\hat{x} + 5\hat{y} + 2\hat{z} \\
            b \cdot c  & = 1                              \\
            b \times c & = \begin{cases}
                               1, 1, 0 \\
                               1, 0, 1
                           \end{cases}                   \\
                       & = <1, 1, -1>                     \\
                       & = \hat{x} - \hat{y} - \hat{z}
        \end{aligned}
    \end{equation}
}

\Problem 1.3 By applying Pythagoras's theorem (the usual two-dimensional version) twice over, prove that the length $\it{r}$ of a three-dimensional vector $r = (x, y, z)$ statisfies $r^2 = x^2 + y^2 + z^2$

\TheSolution
\begin{equation}
    \begin{aligned}
        h^2 & = x^2 + y^2       \\
        r^2 & = h^2 + z^2       \\
        r^2 & = x^2 + y^2 + z^2 \\
    \end{aligned}
\end{equation}

\Problem Find the angle between a body diagonal of a cube and any of its face diagonals. [$\it{Hint:}$ Choose a cube with side 1 adn with one corner at $\it{O}$ and the opposite corner at the point (1, 1, 1). Write down the vector that represents a body diagonal and another that represents a face diagonal, and then find the angle between them as in Problem 1.4].

\TheSolution
\begin{equation}
    \begin{aligned}
        f_{body}                & = <1,1,1>                            \\
        f_{face}                & = <1,1,0>                            \\
        f_{body} \cdot f_{face} & = 2 = |f_{body}||f_{face}|\cos(\phi) \\
        \sqrt3\sqrt2\cos(\phi)  & = 2                                  \\
        \phi                    & = \arccos(\frac{2}{\sqrt3\sqrt2})
    \end{aligned}
\end{equation}
\begin{center}
    \boxed{35.26\deg}
\end{center}

\Problem Prove that the two definitions of the scalar product $r \cdot s$ as $rs\cos(\phi)$ and $\sum r_i s_i$ are equal. One way to do this is to choose your x-axis along the direction of $\boldsymbol{r}$

\TheSolution
If $r$ lays along $x$, then $\cos(\phi) = 1$ since $\phi = 0$. Since $\sum r_i s_i = \sum r_i \cdot \sum s_i = rs$, the two statements are equivalent. The longer way is to use the law of cosines to expand $\cos(\phi)$ in terms of r and s, which will give you an expression that evaluates to a summation.

\Problem
In elementary trigonometry, you probably learned the law of cosines for a triangle of sides $a, b, c$ that $c^2 = a^2 + b^2 - 2ab\cos(\phi)$ where $\phi$ is the angle between the sides $a$ and $b$. Show that the law of cosines is an immediate consequence of the identity $(a + b)^2 = a^2 + b^2 + 2 a \cdot b$.

\TheSolution
\begin{equation}
    \begin{aligned}
        2a \cdot b & = 2ab\cos(\phi) \\
    \end{aligned}
\end{equation}
Since $\phi$ represents the angle between a and b (which is the external angle of the triangle $\pi - \phi$), then $\cos(\phi) \rightarrow -\cos(\phi)$. Let $c = a + b$, then we get $c^2 = a^2 + b^2 - 2ab\cos(\phi)$.

\Problem
The position of a moving particle is given as a function of time $t$ to be
\[r(t) = \hat{x}b\cos(\omega t) + \hat{y}c\sin(\omega t) + \hat{z}v_o t\]
where $b, c, \text{and } \omega$ are constants. Describe the particle's orbit.

\TheSolution
The $\hat{z}v_o t$ part will cause the particle to move upwards continuously, but the two trigonometric functions of different amplitudes will create an elliptical orbit.

\Problem
Let $u$ be an arbitrary fixed unit vector and show that any vector $b$ satisfies
\[b^2 = (u \cdot b)^2 + (u \times b)^2\]

\TheSolution
\begin{equation}
    \begin{aligned}
        b^2 & = (ub\cos(\phi))^2 + (ub\sin(\phi))^2 \\
        b^2 & = (ub)^2(\cos(\phi)^2 + \sin(\phi))^2 \\
        b^2 & = u^2 b^2                             \\
    \end{aligned}
\end{equation}

Since $|u| = 1$ because it is a unit vector, we get $b^2 = b^2$.

\Problem
Show that the definition of the cross product is equivalent to the elementary definition that $r \times s$ is perpendicular to both $r$ and $s$ with magnitude $rs\sin(\phi)$ and direction given by the right-hand rule.

\TheSolution
Let $\vec{v} = (v_1, v_2, v_3)$ and $\vec{s} = (s_1, s_2, s_3)$, then:
\begin{equation}
    \begin{aligned}
        |\vec{v_1}| = \sqrt{v_1^2 + v_2^2 + v_3^2} \\
        |\vec{s_1}| = \sqrt{s_1^2 + s_2^2 + s_3^2} \\
    \end{aligned}
\end{equation}
\begin{equation}
    \begin{aligned}
        |(v_1, v_2, v_3) \times (s_1, s_2, s_3)| & = |(v_2 s_3 - v_3 s_2, v_1 s_3 - v_3 s_1, v_1 s_2 - v_2 s_1)|                     \\
                                                 & = \sqrt{ (v_2 s_3 - v_3 s_2)^2 + (v_1 s_3 - v_3 s_1) ^2 + (v_1 s_2 - v_2 s_1)^2 } \\
                                                 & = \sqrt{
            \begin{aligned}
                (v_2 s_3) ^2    + 2(v_2 v_3 s_2 s_3) + (v_3 s_2) ^2 \\
                +  (v_1 s_3) ^2 + 2(v_1 v_3 s_1 s_3) + (v_3 s_1) ^2 \\
                +  (v_1 s_2) ^2 + 2(v_1 v_2 s_1 s_2) + (v_1 s_2) ^2
            \end{aligned}
        }                                                                                                                            \\
                                                 & = \sqrt{
            \begin{aligned}
                (v_1^2 + v_2^2 + v_3^2)(s_1^2 + s_2^2 + s_3^2) \\
                - ((v_1 s_1)^2 + (v_2 s_2)^2 + (v_3 s_3)^2 +   \\
                2(v_2 v_3 s_2 s_3 + v_1 v_3 s_1 s_3 + v_1 v_2 s_1 s_2))
            \end{aligned}
        }
    \end{aligned}
\end{equation}
And since $(a + b + c)^2 = a^2 + b^2 + c^2 + 2(ab + ac + bc)$, we get:
\begin{equation}
    \begin{aligned}
          & \sqrt{(v_1^2 + v_2^2 + v_3^2)(s_1^2 + s_2^2 + s_3^2) - (v_1 s_1 + v_2 s_2 + v_3 s_3)^2} \\
        = & \sqrt{||\vec{v}||||\vec{s}|| - (\vec{v} \cdot \vec{s})^2}                               \\
    \end{aligned}
\end{equation}
Since $\vec{v} \cdot \vec{s} = ||\vec{v}||||\vec{s}||\cos(\phi)$, we get:
\begin{equation}
    \begin{aligned}
        = & \sqrt{||\vec{v}||^2||\vec{s}||^2 - (\vec{v} \cdot \vec{s}})^2 \\
        = & \sqrt{||\vec{v}||^2||\vec{s}||^2(1 - \cos^2(\phi)})           \\
        = & \sqrt{||\vec{v}||^2||\vec{s}||^2(\sin^2(\phi)})               \\
        = & \boxed{||\vec{v}||||\vec{s}||(\sin(\phi))}                    \\
    \end{aligned}
\end{equation}

\Problem ($\textbf{a}$) Defining the scalar product $\pmb{r \cdot s}$ by Equation (1.7), $\pmb{r \cdot s} = \sum r_i s_i$, show that Pythagoras's theorem implies that the magnitude of any vector $\textbf{r}$ is $\textit{r} = \sqrt{\pmb{r \cdot r}}$. ($\textbf{b}$) It is clear that the length of a vector does not depend on our choicee of coordinate axes. Thus the result of part (a) guarantees that the scalar product $\pmb{r \cdot r}$, as defined by (1.7), is the same for any choice of orthogonal axes. Use this to prove that $\pmb{r \cdot s}$, as defined by (1.7), is the same for any choice of orthogonal axes. [$\textit{Hint:}$ Consider the length of the vector $\pmb{r + s}$]

\TheSolution
$(\textbf{a})$ \[\textit{r} = \sqrt{\sum r_i^2} \rightarrow \textit{r}^2 = \sum r_i^2\]
$(\textbf{b})$ The length of $\vec{r} + \vec{s}$ is  $\sqrt{\sum (r_i + s_i)^2}$
\begin{equation}
    \begin{aligned}
        |\vec{r} + \vec{s}|^2 & = (r_1 + s_1)^2 + (r_2 + s_2)^2 + (r_3 + s_3)^2               \\
                              & = ||\vec{r}||^2||\vec{s}||^2 + 2(r_1 s_1 + r_2 s_2 + r_3 s_3) \\
                              & = ||\vec{r}||^2||\vec{s}||^2 + 2(\vec{r} \cdot \vec{s})       \\
    \end{aligned}
\end{equation}
\[\boxed{\vec{r} \cdot \vec{s} = \frac{1}{2}(|\vec{r} + \vec{s}|^2 - ||\vec{r}||^2||\vec{s}||^2)}\]
Which depends only on the lengths of $\vec{r} \text{ and } \vec{s}$


\Problem (a) Prove that the vector product $r \times s$ as defined by (1.9) is distributive; that is, that $r \times (u + v) = (r \times u) + (r \times v)$. (b) Prove the product rule:
\[ \frac{d}{dt} (r \times s) = r \times \frac{ds}{dt} + \frac{dr}{dt} \times s  \]

\TheSolution
(a)
\begin{equation}
    \begin{aligned}
        r \times (u + v) & = (r_1, r_2, r_3> \times <u_1 + v_1, u_2 + v_2, u_3 + v_3)                                                                           \\
                         & = (r_2(u_3 + v_3) - r_3(u_2 + v_2), r_1(u_3 + v_3) - r_3(u_1 + v_1), r_2(u_3 + v_3) - r_3(u_2 + v_2))                                \\
                         & = (r_2 u_3 + r_2 v_3 - r_3 u_2 - r_3 v_2, r_1 u_3 + r_1 v_3 - r_3 u_1 - r_3 v_1, r_2 u_3 + r_2 v_3 - r_3 u_2 - r_3 v_2)              \\
                         & = ((r_2 u_3 -  r_3 u_2) + (r_2 v_3 - r_3 v_2), (r_1 u_3 - r_3 u_1) + (r_1 v_3 - r_3 v_1), (r_2 u_3 - r_3 u_2) + (r_3 u_2 - r_3 v_2)) \\
                         & \boxed{ = r \times u + r \times v}
    \end{aligned}
\end{equation}
(b)
\begin{equation}
    \begin{aligned}
        \frac{d}{dt}(r \times s) & = \frac{d}{dt}(r_2 s_3 - r_3 s_2, r_1 s_3 - r_3 s_1, r_1 s_2 - r_2 s_1)                     \\
                                 & = \begin{pmatrix}
                                         (r_2 \frac{ds_3}{dt} + \frac{dr_2}{dt} s_3) - (r_3 \frac{ds_2}{dt} + \frac{dr_3}{dt} s_2) \\
                                         (r_1 \frac{ds_3}{dt} + \frac{dr_1}{dt} s_3) - (r_3 \frac{ds_1}{dt} + \frac{dr_3}{dt} s_1) \\
                                         (r_1 \frac{ds_2}{dt} + \frac{dr_1}{dt} s_2) - (r_2 \frac{ds_1}{dt} + \frac{dr_2}{dt} s_1) \\
                                     \end{pmatrix} \\
                                 & = \begin{pmatrix}
                                         (r_2 \frac{ds_3}{dt} - \frac{ds_2}{dt} r_3) + (s_3 \frac{dr_2}{dt} - \frac{dr_3}{dt} s_2) \\
                                         (r_1 \frac{ds_3}{dt} - \frac{ds_1}{dt} r_3) + (s_3 \frac{dr_1}{dt} - \frac{dr_3}{dt} s_1) \\
                                         (r_1 \frac{ds_2}{dt} - \frac{ds_1}{dt} r_2) + (s_2 \frac{dr_1}{dt} - \frac{dr_2}{dt} s_1) \\
                                     \end{pmatrix} \\
                                 & \boxed{= r \times \frac{ds}{dt} + s \times \frac{dr}{dt}}
    \end{aligned}
\end{equation}

\subsection{1.4 Netwon's First and Second Laws; Inertial Frames}
\Problem In case you haven't studied differentia equations before, I shall be introducing the necessary ideas as needed. Here is a simple exercise to get started: Find the general solution of the first-order equation $\frac{df}{dt} = f$ for an unknown function $f(t)$.

\TheSolution
\begin{equation}
    \begin{aligned}
        \frac{df}{dt}     & = f         \\
        \frac{df}{f}      & = dt        \\
        \int \frac{df}{f} & = \int dt   \\
        \ln(f)            & = t + C     \\
        f                 & = e^{t + C} \\
        f                 & = e^t e^C
    \end{aligned}
\end{equation}
One arbitrary variable ($e^C$)

\Problem Answer the same questions as in Problem 2.2.1 but for the differential eqation $\frac{df}{dt} = -3f$.

\TheSolution
\begin{equation}
    \begin{aligned}
        \frac{df}{dt}        & = -3f          \\
        \frac{df}{dt}        & = -3f          \\
        \int \frac{1}{f}  df & = -3 dt        \\
        ln(f)                & = -3t + C      \\
        f                    & = e^{-3t + C}  \\
        f                    & = e^{-3t}e^{C} \\
    \end{aligned}
\end{equation}

\Problem The hallmark of an inertial reference frame is that any object which is subject to zero net force will travel in a straight line at constant speed. To illustrate this, consider the following: I am standing on a level floor at the origin of an inertial frame $\mathcal{S}$ and kick a frictionless puck due north across the floor. (a) Write down the x and y coordinates of the puck as functions of time as seen from my inertial frame. (Use x and y axes pointing east and north respectively.) Now consider two more observers, the first at rest in a frame $\mathcal{S'}$ that travels with cosntant velocity v due east relative to $\mathcal{S}$, the second at rest in frame $\mathcal{S''}$ that travels with constant acceleration due east relative to $\mathcal{S}$. (All three frames coincide at the moment when I kick the puch, and $\mathcal{S''}$ is at rest relative to $\mathcal{S}$ at that same moment.) (b) Find the coordinates x', y' of the puck and describe the puck's path as seen from $\mathcal{S'}$. (c) Do the same for $\mathcal{S''}$. Which of the frames is inertial?

\TheSolution
(a)
\begin{equation}
    \begin{aligned}
        x(t) & = 0                 \\
        y(t) & = \frac{1}{2} a t^2
    \end{aligned}
\end{equation}

(b)
\begin{equation}
    \begin{aligned}
        x(t) & = -v_{\mathcal{S}}(t) \\
        y(t) & = \frac{1}{2} a t^2
    \end{aligned}
\end{equation}

(c)
\begin{equation}
    \begin{aligned}
        x(t) & = -\frac{1}{2}a_{\mathcal{S}}t^2 \\
        y(t) & = \frac{1}{2} a t^2
    \end{aligned}
\end{equation}

(a) and (b) are inertial frames since they aren't undergoing any acceleration.

\Problem I am standing on the ground (which we shall take to be an inertial frame) beside a perfectly flat horizontal turntable, rotating with constant angular velocity $\omega$. I lean over and shove a frictionless puck so that it slides across the turntable, straight through the center. The puck is subject to zero net force so that it slides across the turntable, straight through the center. The puck is subject to zero net force and, as seen from my inertial frame, travels in a straight line. Describe the puck's path as observed by someone sitting at rest on the turntable.

\TheSolution
First, as the observer approaches $\pi$, the puck will look like it isn't travelling very far/fast relative to the observer since they are going the the same direction. Once the observer and puck start going in opposite directions, the puck will look like it is travelling away fairly quickly. The puck will appear to be travelling in somewhat of a spiral shape, curving away from the observer.

\subsection{1.5 The Third Law and Conservation of Momentum}

\Problem Go over the steps from Equation (1.25) to (1.29) in the proof of conservation of momentum, but treat the case that N = 3 and write out all of the summations explicity to be sure you understant the various manipulations.

\TheSolution In this case of N = 3, we have 3 particles that exert some forces on each other but operate in the same inertial frame of reference. Since that is the case, $\dot{p} = F$. The force applied on particle 1 is:
\begin{equation}
    \begin{aligned}
        \sum \vec{F}_1 & = \vec{F}_12 + \vec{F}_13 + \vec{F}_{ext} \\
                       & = \vec{F}_12 + \vec{F}_13 + \vec{F}_{ext}
    \end{aligned}
\end{equation}

Accordingly,
\begin{equation}
    \begin{aligned}
        \sum \vec{F}_2 & = \vec{F}_21 + \vec{F}_23 + \vec{F}_{ext} \\
        \sum \vec{F}_3 & = \vec{F}_31 + \vec{F}_32 + \vec{F}_{ext} \\
    \end{aligned}
\end{equation}

The net force exerted on the system is the sum of the forces experienced by each of the particles. Since $\vec{F}_{ab} = -\vec{F}_{ba}$, we get:
\begin{equation}
    \begin{aligned}
        \sum \vec{F}_{sys}  & = \vec{F}_{ext} \\
        \vec{\dot{p}}_{sys} & = \vec{F}_{ext} \\
    \end{aligned}
\end{equation}

If $\vec{F}_{ext} = 0$ (there is no external force), there is no change in momentum so it is conserved.

\Problem
Conservation laws, such as conservation of momentum, often give a surprising amount of information about the possible outcome of an experiment. Here is perhaps the simplest example: Two objects of masses $m_1$ and $m_2$ are subject to no external forces. Object 1 is traveling with velocity $\textbf{v}$ when it collides with the stationary object 2. The two objects stick together and move off with common velocity $\textbf{v'}$. Use conservation of momentum to find $\textbf{v'}$ in terms of $\textbf{v}, \textit{m}_1, \text{and}, \textit{m}_2$.

\TheSolution Since conservation of momentum applies,
\begin{equation}
    \begin{aligned}
        \sum p_{initial} & = \sum p_{final}          \\
        m_1 v            & = (m_1 + m_2) v'          \\
        v'               & = \frac{m_1 v}{m_1 + m_2}
    \end{aligned}
\end{equation}

\Problem
In section 1.5 we proved that Newton's third law implies the conservation of momentum. Prove the converse, that if the law of conservation of momentum applies to every possible group of particles, then the interparticle forcees must obey the third law.

\TheSolution
If, the conservation of momentum applies, then:
\[\sum \vec{p}_{initial} = \sum \vec{p}_{final} \]
Assuming that there are no external forces acting on the system, we get:
\begin{equation}
    \begin{aligned}
        \vec{0} & = \sum \vec{p}_{final} - \sum \vec{p}_{initial} \\
                & = \Delta \sum \vec{p}                           \\
                & = \vec{\dot{p}}_{sys}                           \\
                & = \vec{F}_{sys}                                 \\
    \end{aligned}
\end{equation}
So the system experiences no internal forces. For the base case of a system with two particles (N = 2), $\vec{F}_{sys} = \vec{F}_{12} + \vec{F}_{21} = 0$, which means that $\vec{F}_{12} = -\vec{F}_{21}$, proof of Newton's Third Law.

\Problem Prove that in the absence of external forces, the total angular momentum (defined as $L = \sum_{\alpha} r_\alpha \times p_\alpha$) of an N-particle system is conserved.

\TheSolution
We are trying to prove that $\vec{L}_sys = \vec{0}$ about a fixed point. Using $\vec{L} = \vec{p} \times \vec{r}$ and differentiating, we get the following:
\begin{equation}
    \begin{aligned}
        \vec{L}_{sys}             & = \sum_i \vec{p}_i \times \vec{r}_i                                                      \\
                                  & = \sum_i (m_i \vec{v}_i) \times \vec{r}_i                                                \\
        \frac{d\vec{L}_{sys}}{dt} & = \sum_i \frac{d}{dt}(m_i \vec{v}_i) \times \vec{r}_i + (m_i \vec{v}_i \times \dot{r}_i) \\
    \end{aligned}
\end{equation}
Since $m_i \vec{v}_i \times \dot{r}_i = 0$ because $\vec{v}_i$ is collinear with $\dot{r}_i$ and $\vec{f}_i \times r_i = -\vec{f}_j \times r_j$. From this, we get that the angular momentum of the system is only based on the moment exerted by external objects.
\end{document}